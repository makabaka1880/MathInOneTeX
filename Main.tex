\documentclass{article}
\usepackage{hyperref}  
\usepackage{graphicx}
\usepackage{amsmath}

\title{Math In One TEX}

\begin{document}
    \begin{titlepage}
        \vspace*{\stretch{1}}
        \begin{center}

            \includegraphics[scale=0.4]{CoverArt.png} \\
            \vspace{4ex}
            \huge{\textsf{Just basicly a open-sourced, TeX-Based documentation of lots of math}}
        \end{center}
        \vspace{\stretch{2}}
    \end{titlepage}


    \clearpage
    \tableofcontents

    % Sec. 1
    \clearpage
    \section{Introduction}
        Math In One TEX aims to document as much mathematic knowledge in one TeX file (.tex) as possible. It is completely open-sourced, and all are welcome to modify and improve.
    
    
    % Sec. 2
    \clearpage
    \section{Basic Constant Arithmatic}
    This section covers basic arithmatic operations that is applied to most constants.


    % Sec. 2 Subsec. 1
    \subsection{Addition}
    \label{section:addition}

    % Sec. 2 Subsec. 1 Subsubsec. 1
    \subsubsection{Definition}
    \par
    Addition of two or more value "results in the total amount or sum of those values combined."\cite{addition} It is the first operation, way back the stone age. There's a long ago story that tells of a shepherd who invented addtion for counting sheeps.
    
    % Sec. 2 Subsec. 1 Subsubsec. 2
    \subsubsection{Pronunciation \& Notation}
    $$a + b + c + ... $$
    \par
    The example above can be pronouced as:
    \begin{itemize}
        \item a plus b plus c plus ...
        \item The sum of a, b, c, ...
    \end{itemize}

    % Sec. 2 Subsec. 1 Subsubsec. 3
    \subsubsection{Laws \& Principles}
    \begin{align*}
        a + b &= b + a &\text{Commutative Law of Addition}\\
        a + (b + c) &= a + b + c &\text{Associative Law of Addition}
    \end{align*}

    % Sec. 2 Subsec. 2
    \subsection{Subtraction}
    \label{section:subtraction}

    % Sec. 2 Subsec. 2 Subsubsec. 1
    \subsubsection{Definition}
    \par
    Subtraction "represents the operation of removing objects from a collection.", and "can also represent removing or decreasing physical and abstract quantities." \cite{subtraction} It is the reverse operation of \nameref{section:addition}.
    \par

    % Sec. 2 Subsec. 2 Subsubsec. 2
    \subsubsection{Pronunciation \& Notation}
    $$a - b - c - ... $$
    The example above can be pronouced as:
    \begin{itemize}
        \item a minus b minus c minus ...
        \item a subtracted by b subtracted by c subtracted by ...
        \item c less then b less then a...
    \end{itemize}

    % Sec. 2 Subsec. 2 Subsubsec. 3
    \subsubsection{Laws \& Principles}
    \begin{align*}
        a - b &= a + (-b) &\text{Basic Principle of Subtraction} \\
        a - b &= -(b - a) &\text{Opposite of s Subtraction Polynomial} \\
    \end{align*}

    % Sec. 2 Subsec. 3
    \subsection{Multiplication}
    \label{section:multiplication}

    % Sec. 2 Subsec. 3 Subsubsec. 1
    \subsubsection{Definition}
    \par
    Multiplication derived from addition. Consider 
    $$ a \times b $$
    \par
    It means 
    $$ \underbrace{a + a + a + ...}_{b\ a's}$$
    \par
    Similarly
    $$ a \times b \times c \times .... $$
    \par
    Means
    $$ \underbrace{\underbrace{a + a + a + ...}_{b\ a's} + \underbrace{a + a + a + ...}_{b\ a's} + \underbrace{a + a + a + ...}_{b\ a's}}_{c\ (a \times b)}$$
    \par

    % Sec. 2 Subsec. 3 Subsubsec. 1
    \subsubsection{Pronunciation \& Notation}
    $$a \times b \times c \times ... $$
    The example above can be pronouced as:
    \begin{itemize}
        \item a times b times c
    \end{itemize}
    The example above can be notated as:
    \begin{itemize}
        \item $a \times b \times c \times ...$
        \item $a \cdot b \cdot c \cdot ...$
        \item $abc...$
    \end{itemize}


    % Sec. 2 Subsec. 3 Subsubsec. 3
    \subsubsection{Laws \& Principles}
    \begin{align*}
        a \cdot b &= b \cdot a &\text{Commutative Law of Multiplication}\\
        a(b \cdot c) &= a \cdot b \cdot c &\text{Associative Law of Multiplication}
    \end{align*}

    % Sec. 2 Subsec. 4
    \subsection{Division}
    \label{section:divison}

    % Sec. 2 Subsec. 4 Subsubsec. 1
    \subsubsection{Definition}
    \par
    Division is the reverse operation of \nameref{section:multiplication}. It is "the process of calculating the number of times one number is contained within another."\cite{division} Consider
    $$ a \cdot b = c$$
    We can infer from the equation that b time a's value equals to c. Therefore, we can infer that
    $$
    \left\{
        \begin{aligned}
            c \div b = a\\
            c \div a = b
        \end{aligned}
    \right.
    $$
    \par

    % Sec. 2 Subsec. 4 Subsubsec. 1
    \subsubsection{Pronunciation \& Notation}
    $$a \div b \div c \div ... $$
    The example above can be pronouced as:
    \begin{itemize}
        \item a divided by b divided by c ...
        \item one $\mathrm c^{th}$ of one $\mathrm b^{th}$ of one $\mathrm a^{th}$ of ...
    \end{itemize}
    The example above can be notated as:
    \begin{itemize}
        \item $a \div b \div c \div ...$
        \item $\frac{a}{\frac{b}{\frac{c}{...}}}$
    \end{itemize}


    % Sec. 2 Subsec. 4 Subsubsec. 3
    \subsubsection{Laws \& Principles}
    \begin{align*}
        a \div b &= a \cdot \frac{1}{b} &\text{Basic Principle of Division}\\
        a \div b &= \frac{1}{a \div b} &\text{Reciprocal of a Rational Expression}
    \end{align*}

    \clearpage
    \bibliographystyle{apalike}
    \bibliography{Citations}

\end{document}

